% Lab 1 LaTeX Document
% Version 1.0
\documentclass{article}
\usepackage{listings}
\usepackage[margin=1in]{geometry}
\usepackage{graphicx}
\usepackage{textcomp}
\usepackage[pdftex,
            pdfauthor={Daniel Noyes},
            pdftitle={Laboratory \#1 - Spring 2016 },
            pdfsubject={ECE 368 Digital Design},
            pdfkeywords={Digital Design, VHDL, Nexys 2, Xilinx, ISE},
            pdfproducer={Latex},
            pdfcreator={pdflatex}]{hyperref}

%Remove the number display for each section tag
\makeatletter
\renewcommand\@seccntformat[1]{}
\makeatother

\begin{document}

\begin{center}
\textsc{\huge ECE 368 Digital Design - Spring 2016}\\[1cm]
\textsc{{\LARGE Laboratory \#1: (100pts)}}\\[0.5cm]
\textsc{\Large Lab date: January $29$\textsuperscript{th},2016}\\[0.5cm]
\textsc{\Large Lab Report Due: Friday February $5$\textsuperscript{th},2016}\\[1cm]
\end{center}

\section{Overview and Objectives:}
In this laboratory assignment, you will be using the Xilinx ISE Design Suite to develop hardware using VHDL. The main object for the lab is to learn fundamental concepts and using the Xilinx ISE tools to help aid in the development of hardware design. You will be able to understand the principles of checking syntax, understanding various syntax issues, generate a register-transfer level (RTL) and a technology schematic as well. You will also examine the use of test bench's for design verification to validate that the design meets the set standards for the project.

\section{Objectives for this lab:}
\begin{itemize}
  \item An introduction to understanding VHDL code
  \item Understand a finite state machine (FSM)
  \item In depth understanding of a Mealy FSM
  \item Understand basic ISE warning/error messages
  \item Learn how to create a register-transfer level from VHDL source
  \item Able to read and understand RTL diagrams
  \item Learn how to create a technology schematic and see what the design uses
  \item Learn how to create a test bench and run a simulation
  \item Able to assemble all the concepts together
\end{itemize}

\section{Lab sections:}
\begin{enumerate}
  \item Introduction to Xilinx ISE
  \item Understanding Register-Transfer Level
  \item Learning about Technology Schematics
  \item Simulating for the first time
\end{enumerate}

\newpage

\section{1. Introduction to Xilinx ISE:}
Welcome to the introduction to Xilinx. You probably are wondering what is Xilinx ISE. Xilinx ISE (Integrated Synthesis Environment) is a tool to analyze Hardware descriptive language designs and synthesizes them to Xilinx FPGA's. In this lab you will be using VHDL for all your hardware description language.

\subsection{Creating your first Xilinx Project}

You will need the mealy project code which can be found \href{https://github.com/reiuiji/ECE368-Lab/tree/master/Lab%201/MealyFSM/}{here} or on the class directory.
Now lets start ISE project navigator. When you open up Xilinx ISE project navigator, you will be greater to the welcome screen. You need to create a new project, you can do this either from the welcome window or \textbf{File -\textgreater new project}.

\begin{figure}[!h]
  \centering
    \fbox{\includegraphics[width=0.5\textwidth]{images/XilinxNewProject.png}}
  \caption{ISE\textsuperscript{\textregistered} Design suite welcome page}
\end{figure}

\textbf{Steps for creating a new project in ISE:}
\begin{enumerate}
  \item Select \textbf{File -\textgreater new project} to open the New Project Wizard.
  \item Enter a name for the project, for this project is will be \textbf{MealyTest}
  \item Enter a location where the project will be located
  \item Enter a description for the project
  \item Select \textbf{HDL} in the top-level source type
  \item Click next to advance to the project setting page
  \item Enter the parameters for the project: \hfill \\
  Family : \textbf{Spartan3E}\\
  Device : \textbf{XC3S500E}\\
  Package : \textbf{FG320}\\
  Speed  : \textbf{-4}\\
  Systhesis Tool : \textbf{XST (VHDL/Verilog)}\\
  Simulator : \textbf{ISim (VHDL/Verilog)}\\
  Preferred Language : \textbf{VHDL}\\
  Property Specification in Project File : \textbf{Store all values}\\
  Uncheck Manual Compile Order\\
  VHDL Source Analysis Standard : \textbf{VHDL-93}\\
  Uncheck Message Filtering
  \item Double Check the Project Setting with the Setting Figure
\end{enumerate}

\newpage %temporary fix to get the figures align next to the setup

\begin{figure}[!htb]
  \centering
    \fbox{\includegraphics[width=0.6\textwidth]{images/NewProjectWizard1.png}}
  \caption{New Project Wizard Create New Project page}
\end{figure}

\begin{figure}[!htb]
  \centering
    \fbox{\includegraphics[width=0.6\textwidth]{images/NewProjectWizard2.png}}
  \caption{New Project Wizard Project Settings page}
\end{figure}

\subsection{Adding VHDL Source to the Project}
After you have created the Project in ISE. You now need to import the mealy source files into it. This can be done by going to \textbf{Project -\textgreater Add Source}. Select the two files in the MealyFSM folder. After you imported all the files needed you will be presented with a selection for implementation file on the side of Xilinx ISE.

\subsection{Building the mealy project}
Now since you have added the necessary source files required for the Mealy FSM it is time to check if it would run. To check the syntax of the VHDL source code you would need to run the "Check Syntax" operation under the "Synthesize - XST" folder. During this process is where you will encounter various warnings and errors that may occur while validating the VHDL source code for the project.

\begin{figure}[!htb]
  \centering
    \fbox{\includegraphics[width=0.5\textwidth]{images/XilinxISEImplementationMenu.png}}
  \caption{Synthesize options in ISE}
\end{figure}


\subsection{Syntax issues and resolution}
While checking the syntax of the VHDL source could you will be seeing one warning and one error. The error is tied with a state machine conflict with the system. The warning will tell you that the "Process" will run on infinite and will need two signals to process on.

\begin{figure}[!htb]
  \centering
    \fbox{\includegraphics[width=0.5\textwidth]{images/XilinxErrorInfo.png}}
  \caption{Error \& Warning Messages from running "Check Syntax" Operation}
\end{figure}

\subsection{Introduction Questions}
Looking through the previous syntax issues. What were the causes for the warning and error. How did you resolve the issue? Make a note in your lab report on your findings and steps taken to fix the problem.

\newpage

\section{2. Understanding Register-transfer level:}
Under Xilinx ISE you have the ability to generate a register-transfer level which will help aid in understanding the overall structure of the design.

\subsection{Generating a Register-transfer level schematic}
In order to view the RTL Schematic for the project you have to run "View RTL Schematic" under the "Synthesize - XST" Folder. When the process is completed an RTL/Tech Viewer will open to ask if you want to open up the RTL in the Explorer Wizard or the schematic of the top-level block. Select the top-level block and click "OK". You will now be able to see the overall RTL of the system. 

\textbf{HINT:} If you have issues with viewing a RTL make sure the entity you are reviewing will have both input and output connections tied out. Xilinx ISE will trim your project of any block that have no input and output since it will not affect the overall system.

\begin{figure}[!htb]
  \centering
    \fbox{\includegraphics[width=0.7\textwidth]{images/XilinxRTLView.png}}
  \caption{Xilinx RTL/Tech Viewer Startup Mode Menu }
\end{figure}

\subsection{RTL Questions}
After you have viewed the RTL of the top-level block. Open up the block and look inside to see what components were used to build the Mealy FSM. Once you are done take a screen shot of the RTL and attach it to the lab report.

\newpage

\section{3. Learning about Technology schematics:}
Besides reviewing the RTL schematic, you also have the ability to see the overall technology schematic. The technology schematic shows what the actual hardware will be implemented for the design onto the FPGA. You will be able to view the technology schematic the same way as the RTL under the "Synthesize - XST" Folder.

\subsection{Technology schematics Questions:}
After you have viewed the Technology schematic of the top-level block. Open up the block and look inside to see what hardware components were used to build the Mealy FSM on a FPGA. Once you are done take a screen shot of the technology schematic and attach it to the lab report.

\section{4. Simulating for the first time}
Once you have analyzed the structure of you design. You will now need to test your design against the use case for the project. When you will be working in the real world, you will have to be testing your design to meet the customers needs for the project. This is a common way to develop a product with the necessary functionality for a given contract.

In order to run a simulation of your code you will have to switch your design view from "Implementation" to "Simulation" as shown in the next figure. When you are in simulation mode your run options will change. You will now need to import the "mealy-testbench.vhd" into the project to run a test bench. When you have the test bench in the project. Select the "mealy\_tb" under the hierarchy menu and click "Simulate Behavior Model". If you accident run the simulation under mealy you will not be able to see any data in the simulation since no data around the chip will be changing.

\begin{figure}[!htb]
  \centering
    \fbox{\includegraphics[width=0.5\textwidth]{images/XilinxDesignView.png}}
  \caption{ISE Design View Option}
\end{figure}

\begin{figure}[!htb]
  \centering
    \fbox{\includegraphics[width=0.5\textwidth]{images/XilinxSimulationMenu.png}}
  \caption{ISim Simulator Options}
\end{figure}

Once the "Simulate Behavior Model" is complete a new window will open up. This is the ISim utility which will help debug your design and will provide all the events that are happening in you design. When the simulation finish loading you will see a wave form generated from the signals you have created in the test bench. On the top right of the simulation window you will see various time options. These options will allow you to extend the simulation or restart it if need be.

\begin{figure}[!htb]
  \centering
    \fbox{\includegraphics[width=0.5\textwidth]{images/XilinxISimTimebar.png}}
  \caption{ISim Time options}
\end{figure}

\textbf{HINT:} If you need to grab other processes to monitor you can open up the entity on the "Instances and Processes" window on the left. When you grab the signal you want to watch go and right click and click on "Add to Wave Window". The simulation will show these new signals as blank since ISim did not record the data. On the time bar click on "restart" then click "run for the time specified". Now you will be able to see the newly monitored data.

\begin{figure}[!htb]
  \centering
    \fbox{\includegraphics[width=0.5\textwidth]{images/XilinxISimProcesses.png}}
  \caption{ISim Process view}
\end{figure}

\subsection{Simulation Questions:}

Under the simulation you will need to collect various information and see if your design operates to your specifications.

Complete the following Questions:

\begin{enumerate}
  \item Take a screen shot of the wave form from 100ns to 300ns.
  \item Take a screen shot for the next time "z\_out\_test" goes high.
  \item What happens after problem 2? Does it continue?
  \item what will be the state of the signals Current State \& Next State before, during, and after "z\_out\_test" goes high at 260ns - 270ns?
\end{enumerate}


\newpage

\section{Lab 1 Grade}

\begin{table}[!htb]
  \begin{center}
    \begin{tabular}[width=0.8\textwidth]{|l|c|l|}
       \hline
       Section & Value & Score\\
       \hline 
       \multicolumn{2}{|l}{\textbf{Section 1}} &\\
       \hline
       1.1. Introduction Questions & 20 &\\
       \hline
       \multicolumn{2}{|l}{\textbf{Section 2}}  &\\
       \hline
       2.1. RTL Questions & 10 &\\
       \hline
       \multicolumn{2}{|l}{\textbf{Section 3}}  &\\
       \hline
       3.1. Technology schematics Questions: & 10 &\\
       \hline
       \multicolumn{2}{|l}{\textbf{Section 4}}  &\\
       \hline
       4.1. Simulation Questions: & 10 &\\
       \hline
       \multicolumn{1}{|l}{\textbf{Lab Report}}  & -50- &\\
       \hline
       \hline
       \multicolumn{1}{|l}{Total} & \multicolumn{1}{c|}{100} &\\
       \hline
    \end{tabular}
  \end{center}
  \caption{Lab Grade Breakdown Table}
\end{table}

\end{document}
