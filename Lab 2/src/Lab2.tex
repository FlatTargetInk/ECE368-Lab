% Lab 2 LaTeX Document
% Version 1.0
\documentclass{article}
\usepackage{listings}
\usepackage[margin=1in]{geometry}
\usepackage{graphicx}
\usepackage{textcomp}
\usepackage{xcolor,colortbl}
\usepackage{longtable}
\usepackage{caption}
\usepackage[pdftex,
            pdfauthor={Daniel Noyes},
            pdftitle={Laboratory \#2 - Spring 2015 },
            pdfsubject={ECE 368 Digital Design},
            pdfkeywords={Digital Design, VHDL, Nexys 2, Xilinx, ISE},
            pdfproducer={Latex},
            pdfcreator={pdflatex}]{hyperref}


\captionsetup[table]{justification=centering}

%Remove the number display for each section tag
\makeatletter
\renewcommand\@seccntformat[1]{}
\makeatother

%Define gray center for a table
\definecolor{Gray}{gray}{0.85}
\newcolumntype{g}{>{\columncolor{Gray}}c}
%Quick command for multiple lines in a table cell
\newcommand{\specialcell}[2][c]{%
  \begin{tabular}[#1]{@{}c@{}}#2\end{tabular}}


\begin{document}

\begin{center}
\textsc{\huge ECE 368 Digital Design - Spring 2015}\\[1cm]
\textsc{{\LARGE Laboratory \#2: (100pts)}}\\[0.5cm]
\textsc{\Large Lab date: February $11$\textsuperscript{th},2015}\\[0.5cm]
\textsc{\Large Lab Report Due: Wednesday February $25$\textsuperscript{th},2015}\\[1cm]
\end{center}

\section{Overview and Objectives:}
In this laboratory assignment, you will build upon from you previous laboratory assignment and dive deeper into VHDL. You will be able to understand IBM's Personal System 2 (PS/2) protocol. You will learn how to convert from a PS/2 input signal directly to an ascii representation. Next you will be learning the process of outputting data to a VGA display. With the VGA display, you will then learn how to create a buffer to hold characters to display on the VGA for a basic terminal. Last you will be introduced to a debug unit which will aid you in testing your design in the project.

\section{Objectives for this lab:}
\begin{itemize}
  \item Understand the PS/2 Protocol and handle keyboard input.
  \item Learn how to convert from PS/2 \textbf{keycode} to \textbf{ascii} with lookup tables.
  \item Learn the process of testing designs in simulation.
  \item Experience in using a Finite State Machine (FSM) design concepts.
  \item Understand the process of displaying to a VGA display.
  \item Able to change the VGA display color on the screen.
  \item Learn how to create a memory buffer for a VGA display.
  \item Understand how to generate block RAM with Xilinx\textsuperscript{\textregistered} logicore utility.
  \item Learn how to code yourself a debug unit.
  \item Able to assemble previous concepts together.
\end{itemize}

\section{Lab sections:}
\begin{enumerate}
  \item Introduction with a keyboard.
  \item VGA Display concepts part 1.
  \item VGA Display concepts part 2.
  \item Debug Unit.
  \item Assembling a Keyboard Debug Unit.
  \item Extra 1: Keyboard ALU part 1
  \item Extra 2: Keyboard ALU part 2
\end{enumerate}

\newpage
\section{1. Introduction with a keyboard:}
In this section of the lab you will go though the concepts of a PS/2. You will be able to build a PS/2 keyboard interface and have it output the ascii character code on the 8 LED's on the Nexys board. You will also be analyzing a test bench of the keyboard controller to gain a better understanding of the concept.

\subsection{PS/2 Keyboard Concepts}
The PS/2 controller described in this lab can be used as a bi-directional communication device to receive and transmit information between the keyboard and the PS/2 controller. For the basics on the PS/2 protocol, the protocol uses one clock line and one data line. The data can transmit back and forth between the keyboard and the controller in serial. When the clock line goes from high to low, it tells the controller there is data available. The controller will grab the data and wait for the next bit to arrive. The data is transmitted in a 11 bit packet with a start bit, 8 bit message, parity, and a stop bit. The controller will check if the data is valid though the parity bit.

\begin{figure}[!htbp]
  \centering
    \fbox{\includegraphics[width=0.8\textwidth]{images/PS2Face.png}}
  \caption{PS/2 Layout}
\end{figure}

The method in which the data being transmitted from the PS/2 keyboard is through scan codes. When you press any key on the keyboard, a unique code will be sent. The keyboard will send a \textbf{X"F0"} before the scan code when a key is released. When a extended key is pressed(arrow keys, right ctrl/Alt) on the keyboard, the keyboard will send out three scan codes. The scan codes go in the order \textbf{X"E0"} \textbf{X"F0"} followed by the scan code. For example if you press the letter \textbf{a}, the keyboard will send out \textbf{X"F0"},\textbf{X"1C"}.

\begin{table}[!htbp]
  \begin{center}
    \begin{tabular}{|l|l|}
       \hline
       \large{\textbf{Command Codes:}} & \large{\textbf{Description:}} \\
       \hline 
       \textbf{X"F0"} & Key has been released \\
       \textbf{X"E0"} & Extended Key code is detected \\
       \textbf{X"FF"} & Tell the keyboard to reset \\
       \textbf{X"ED"} & Set keyboard LED Lock Indicator: Num(bit 0), Caps(bit 1), Scroll(bit 2) \\
       \textbf{X"FA"} & Keyboard Acknowledge for X"ED" \\
       \hline
    \end{tabular}
  \end{center}
  \caption{Common Keyboard Command Codes}
\end{table}

One the next few pages are the tables for the Keyboard Scan Codes and a ASCII Table for reference. For this lab, you will be testing the keyboard controller provided in this lab. A test bench is provided for you to gain a better understanding of what is happening in the code. I would strongly recommend analyzing the code in the \textbf{keycode\_to\_ascii.vhd} source file. In this file there are two lookup tables which will convert the keyboard scan codes directly to ASCII(upper/lower case). The file also houses a finite state machine. This state machine monitors the inputs from the keyboard driver and updates the state of the ASCII based on the order of keycode. The keyboard can only send data to the FPGA if both clock and data lines are set high which indicate the line is idle.

\newpage
%Keyboard Scan Codes
    \begin{longtable}{|c|c|c|g|c|c|c|g|c|c|c|}
       \hline
       \multicolumn{11}{|c|}{{\huge \textbf{Keyboard Scan Codes}}} \\
       \multicolumn{11}{|c|}{{\Large \textbf{All values are in Hexadecimal.}}} \\
       \hline 
       {\small \textbf{KEY}} & {\small \textbf{MAKE}} & {\small \textbf{BREAK}} && 
       {\small \textbf{KEY}} & {\small \textbf{MAKE}} & {\small \textbf{BREAK}} && 
       {\small \textbf{KEY}} & {\small \textbf{MAKE}} & {\small \textbf{BREAK}} \\
       \hline
       A & 1C & F0,1C && 9 & 46 & F0,46 && [ & 54 & F0,54 \\
       B & 32 & F0,32 && ` & 0E & F0,0E && INSERT & E0,70 & E0,F0,70 \\
       C & 21 & F0,21 && - & 4E & F0,4E && HOME & E0,6C & E0,F0,6C \\
       D & 23 & F0,23 && = & 55 & F0,55 && PG UP & E0,7D & E0,F0,7D \\
       E & 24 & F0,24 && \ & 5D & F0,5D && {\small DELETE} & E0,71 & E0,F0,71 \\
       F & 2B & F0,2B && BKSP & 66 & F0,66 && END & E0,69 & E0,F0,69 \\
       G & 34 & F0,34 && SPACE & 29 & F0,29 && PG DN & E0,7A & E0,F0,7A \\
       H & 33 & F0,33 && TAB & 0D & F0,0D && {\small U ARRW} & E0,75 & E0,F0,75 \\
       I & 43 & F0,4C && CAPS & 58 & F0,58 && {\small L ARRW} & E0,6B & E0,F0,6B \\
       J & 3B & F0,3B && L SHFT & 12 & F0,12 && {\small D ARRW} & E0,72 & E0,F0,72 \\
       K & 42 & F0,42 && L CTRL & 14 & F0,14 && {\small R ARRW} & E0,74 & E0,F0,74 \\
       L & 4B & F0,4B && L GUI & E0,1F & E0,F0,1F && NUM & 77 & F0,77 \\
       M & 3A & F0,3A && L ALT & 11 & F0,11 && KP / & E0,4A & E0,F0,4A \\
       N & 31 & F0,31 && R SHFT & 59 & F0,59 && KP * & 7C & F0,7C \\
       O & 44 & F0,44 && R CTRL & E0,14 & F0,14 && KP - & 7B & F0,7B \\
       P & 4D & F0,4D && R GUI & E0,27 & E0,F0,27 && KP + & 79 & F0,79 \\
       Q & 15 & F0,15 && R ALT & E0,11 & E0,F0,11 && KP EN & E0,5A & E0,F0,5A \\
       R & 2D & F0,2D && APPS & E0,2F & E0,F0,2F && KP . & 71 & F0,71 \\
       S & 1B & F0,1B && ENTER & 5A & F0,5A && KP 0 & 70 & F0,70 \\
       T & 2C & F0,2C && ESC & 76 & F0,76 && KP 1 & 69 & F0,69 \\
       U & 3C & F0,3C && F1 & 05 & F0,05 && KP 2 & 72 & F0,72 \\
       V & 2A & F0,2A && F2 & 06 & F0,06 && KP 3 & 7A & F0,7A \\
       W & 1D & F0,1D && F3 & 04 & F0,04 && KP 4 & 6B & F0,6B \\
       X & 22 & F0,22 && F4 & 0C & F0,0C && KP 5 & 73 & F0,73 \\
       Y & 35 & F0,35 && F5 & 03 & F0,03 && KP 6 & 74 & F0,74 \\
       Z & 1A & F0,1A && F6 & 0B & F0,0B && KP 7 & 6C & F0,6C \\
       0 & 45 & F0,45 && F7 & 83 & F0,83 && KP 8 & 75 & F0,75 \\
       1 & 16 & F0,16 && F8 & 0A & F0,0A && KP 9 & 7D & F0,7D \\
       2 & 1E & F0,1E && F9 & 01 & F0,01 && ] & 5B & F0,5B \\
       3 & 26 & F0,26 && F10 & 09 & F0,09 && ; & 4C & F0,4C \\
       4 & 25 & F0,25 && F11 & 78 & F0,78 && ' & 52 & F0,52 \\
       5 & 2E & F0,2E && F12 & 07 & F0,07 && , & 41 & F0,41 \\
       6 & 36 & F0,36 && \specialcell{PRNT\\SCRN} & \specialcell{E0,12,\\E0,7C} & \specialcell{E0,F0,\\7C,E0,\\F0,12} && . & 49 & F0,49 \\
       7 & 3D & F0,3D && {\small SCROLL} & 58 & F0,46 && / & 4A & F0,4A \\
       8 & 3E & F0,3E && PAUSE & \specialcell{E1,14,\\77,E1,\\F0,14,\\F0,77} & -NONE- && & & \\
       \hline
  \caption{Keyboard Scan Code table \\(Source: \href{http://www.computer-engineering.org/ps2keyboard/scancodes2.html}{computer-engineering.org})}
\end{longtable}

\newpage
%ASCII Table
    \begin{longtable}{|c|c|c|c|c|l|}
       \hline
       \multicolumn{6}{|c|}{{\huge \textbf{ASCII Code Table}}} \\
       \hline 
       \textbf{Decimal} & \textbf{Octal} & \textbf{Hex} & \textbf{Binary} & \textbf{Value} & \textbf{Description}\\
       \hline
       000 & 000 & 00 & 00000000 & NUL & Null Char.\\
       001 & 001 & 01 & 00000001 & SOH & Start of Header\\
       002 & 002 & 02 & 00000010 & STX & Start of Text\\
       003 & 003 & 03 & 00000011 & ETX & End of Text\\
       004 & 004 & 04 & 00000100 & EOT & End of Transmission\\
       005 & 005 & 05 & 00000101 & ENQ & Enquiry\\
       006 & 006 & 06 & 00000110 & ACK & Acknowledgment\\
       007 & 007 & 07 & 00000111 & BEL & Bell\\
       008 & 010 & 08 & 00001000 & BS  & Backspace\\
       009 & 011 & 09 & 00001001 & HT  & Horizontal Tab\\
       010 & 012 & 0A & 00001010 & LF  & Line Feed\\
       011 & 013 & 0B & 00001011 & VT  & Vertical Tab\\
       012 & 014 & 0C & 00001100 & FF  & Form Feed\\
       013 & 015 & 0D & 00001101 & CR  & Carriage Return\\
       014 & 016 & 0E & 00001110 & SO  & Shift Out\\
       015 & 017 & 0F & 00001111 & SI  & Shift In\\
       016 & 020 & 10 & 00010000 & DLE & Data Link Escape\\
       017 & 021 & 11 & 00010001 & DC1 & Device Control 1 (XON)\\
       018 & 022 & 12 & 00010010 & DC2 & Device Control 2\\
       019 & 023 & 13 & 00010011 & DC3 & Device Control 3 (XOFF)\\
       020 & 024 & 14 & 00010100 & DC4 & Device Control 4\\
       021 & 025 & 15 & 00010101 & NAK & Negative Acknowledgment\\
       022 & 026 & 16 & 00010110 & SYN & Synchronous Idle\\
       023 & 027 & 17 & 00010111 & ETB & End of Trans. Block\\
       024 & 030 & 18 & 00011000 & CAN & Cancel\\
       025 & 031 & 19 & 00011001 & EM  & End of Medium\\
       026 & 032 & 1A & 00011010 & SUB & Substitute\\
       027 & 033 & 1B & 00011011 & ESC & Escape\\
       028 & 034 & 1C & 00011100 & FS  & File Separator\\
       029 & 035 & 1D & 00011101 & GS  & Group Separator\\
       030 & 036 & 1E & 00011110 & RS  & Request to Send (Rec. Sep.)\\
       031 & 037 & 1F & 00011111 & US  & Unit Separator\\
       032 & 040 & 20 & 00100000 & SP  & Space\\
       033 & 041 & 21 & 00100001 & !   & Exclamation Mark\\
       034 & 042 & 22 & 00100010 & "   & Double Quote\\
       035 & 043 & 23 & 00100011 & \#  & Number Sign\\
       036 & 044 & 24 & 00100100 & \$  & Dollar Sign\\
       037 & 045 & 25 & 00100101 & \%  & Percent\\
       038 & 046 & 26 & 00100110 & \&  & Ampersand\\
       039 & 047 & 27 & 00100111 & '   & Single Quote\\
       040 & 050 & 28 & 00101000 & (   & Left/Opening parenthesis\\
       041 & 051 & 29 & 00101001 & )   & Right/Closing Parenthesis\\
       042 & 052 & 2A & 00101010 & *   & Asterisk\\
       043 & 053 & 2B & 00101011 & +   & Plus\\
       044 & 054 & 2C & 00101100 & ,   & comma\\
       045 & 055 & 2D & 00101101 & -   & Minus/Dash\\
       046 & 056 & 2E & 00101110 & .   & Dot\\
       047 & 057 & 2F & 00101111 & /   & Forward Slash\\
       048 & 060 & 30 & 00110000 & 0   & \\
       049 & 061 & 31 & 00110001 & 1   & \\
       050 & 062 & 32 & 00110010 & 2   & \\
       051 & 063 & 33 & 00110011 & 3   & \\
       052 & 064 & 34 & 00110100 & 4   & \\
       053 & 065 & 35 & 00110101 & 5   & \\
       054 & 066 & 36 & 00110110 & 6   & \\
       055 & 067 & 37 & 00110111 & 7   & \\
       056 & 070 & 38 & 00111000 & 8   & \\
       057 & 071 & 39 & 00111001 & 9   & \\
       058 & 072 & 3A & 00111010 & :   & Colon\\
       059 & 073 & 3B & 00111011 & ;   & Semi-Colon\\
       060 & 074 & 3C & 00111100 & $<$ & Less Than\\
       061 & 075 & 3D & 00111101 & =   & Equal Sign\\
       062 & 076 & 3E & 00111110 & $>$ & Greater Than\\
       063 & 077 & 3F & 00111111 & ?   & Question Mark\\
       
       
       
       064 & 100 & 40 & 01000000 & @   & AT Symbol\\
       065 & 101 & 41 & 01000001 & A   & \\
       066 & 102 & 42 & 01000010 & B   & \\
       067 & 103 & 43 & 01000011 & C   & \\
       068 & 104 & 44 & 01000100 & D   & \\
       069 & 105 & 45 & 01000101 & E   & \\
       070 & 106 & 46 & 01000110 & F   & \\
       071 & 107 & 47 & 01000111 & G   & \\
       072 & 110 & 48 & 01001000 & H   & \\
       073 & 111 & 49 & 01001001 & I   & \\
       074 & 112 & 4A & 01001010 & J   & \\
       075 & 113 & 4B & 01001011 & K   & \\
       076 & 114 & 4C & 01001100 & L   & \\
       077 & 115 & 4D & 01001101 & M   & \\
       078 & 116 & 4E & 01001110 & N   & \\
       079 & 117 & 4F & 01001111 & O   & \\
       080 & 120 & 50 & 01010000 & P   & \\
       081 & 121 & 51 & 01010001 & Q   & \\
       082 & 122 & 52 & 01010010 & R   & \\
       083 & 123 & 53 & 01010011 & S   & \\
       084 & 124 & 54 & 01010100 & T   & \\
       085 & 125 & 55 & 01010101 & U   & \\
       086 & 126 & 56 & 01010110 & V   & \\
       087 & 127 & 57 & 01010111 & W   & \\
       088 & 130 & 58 & 01011000 & X   & \\
       089 & 131 & 59 & 01011001 & Y   & \\
       090 & 132 & 5A & 01011010 & Z   & \\
       091 & 133 & 5B & 01011011 & [   & Left/Opening Bracket\\
       092 & 134 & 5C & 01011100 & \   & Back Slash\\
       093 & 135 & 5D & 01011101 & ]   & Right/Closing Bracket\\
       094 & 136 & 5E & 01011110 & \textasciicircum   & Caret/Circumflex\\
       095 & 137 & 5F & 01011111 & \_   & Underscore\\

       
       096 & 140 & 60 & 01100000 & `   & \\
       097 & 141 & 61 & 01100001 & a   & \\
       098 & 142 & 62 & 01100010 & b   & \\
       099 & 143 & 63 & 01100011 & c   & \\
       100 & 144 & 64 & 01100100 & d   & \\
       101 & 145 & 65 & 01100101 & e   & \\
       102 & 146 & 66 & 01100110 & f   & \\
       103 & 147 & 67 & 01100111 & g   & \\
       104 & 150 & 68 & 01101000 & h   & \\
       105 & 151 & 69 & 01101001 & i   & \\
       106 & 152 & 6A & 01101010 & j   & \\
       107 & 153 & 6B & 01101011 & k   & \\
       108 & 154 & 6C & 01101100 & l   & \\
       109 & 155 & 6D & 01101101 & m   & \\
       110 & 156 & 6E & 01101110 & n   & \\
       111 & 157 & 6F & 01101111 & o   & \\
       112 & 160 & 70 & 01110000 & p   & \\
       113 & 161 & 71 & 01110001 & q   & \\
       114 & 162 & 72 & 01110010 & r   & \\
       115 & 163 & 73 & 01110011 & s   & \\
       116 & 164 & 74 & 01110100 & t   & \\
       117 & 165 & 75 & 01110101 & u   & \\
       118 & 166 & 76 & 01110110 & v   & \\
       119 & 167 & 77 & 01110111 & w   & \\
       120 & 170 & 78 & 01111000 & x   & \\
       121 & 171 & 79 & 01111001 & y   & \\
       122 & 172 & 7A & 01111010 & z   & \\
       123 & 173 & 7B & 01111011 & \{   & Left/Opening Brace\\
       124 & 174 & 7C & 01111100 & |   & Vertical Bar\\
       125 & 175 & 7D & 01111101 & \}   & Right/Closing Brace\\
       126 & 176 & 7E & 01111110 & ~   & Tilde\\
       127 & 177 & 7F & 01111111 & DEL & Delete\\
       \hline
    \caption{Keyboard Scan Code table \\(Source: \href{http://www.computer-engineering.org/ps2keyboard/scancodes2.html}{computer-engineering.org})}
    \end{longtable}

\begin{figure}[!htbp]
  \centering
    \fbox{\includegraphics[width=0.9\textwidth]{images/PS2KCodes.png}}
  \caption{PS/2 Keyboard scan codes}
\end{figure}

\newpage
\subsection{Breakdown of the Keyboard Controller}
With the experience you gain from the previous lab, you now know how to setup a project and grab the vhd files. For the Keyboard Controller, you need to get all six files(vhd,ucf) in the Keyboard folder for this section of the lab. When your looking at each file you can see what they are designed for. I recommend reading the files from the top down to give a perspective of the device. You don't need to look at the PS2 Driver but everything else is a must.

\begin{figure}[!htbp]
  \centering
    \fbox{\includegraphics[width=0.9\textwidth]{images/KeyboardControllerRTL.png}}
  \caption{Keyboard Controller RTL Diagram}
\end{figure}

\subsection{Testing the Keyboard Controller}
After you looked at each file for the keyboard controller. You are now ready to test it out. First you should simulate the design to see how it outputs the ASCII data. There is already a test bench create for this test. All you have to do is have the simulation go from \textbf{1us} to \textbf{2ms}. This can be done on the top right bar on the ISim utility. Go and look at the simulation and capture the wave form when the ASCII value updates to the letter "\textbf{a}". When you zoom into the simulation, you can see \textbf{ASCII\_RD} rise high for one clock cycle. This line is used to tell other devices that there is a new ASCII code on the bus. \\

After you simulated the keyboard controller, you will now need to build and flash it on the Neyxs 2 board in order to test it out. Just like in the real world, you will be needing to test you device throughly. This is normally done with an hardware acceptance test plan(ATP). For this section of the lab you are going to be creating a test plan for the keyboard controller. You will be testing the basic ASCII codes(a-z,A-Z,0-9) to verify that they are outputting correctly. In the keyboard folder there is a Test Plan document to help with testing the basic inputs and outputs of the device. You will be needing to test each output to see what is the result and trace it to the cause of the problem.

\subsection{Keyboard Objectives}
For the Keyboard Controller Section you will be asked several questions and demonstrate a working keyboard controller to the TA or Professor. As stated before, write down what you did during the lab so you can easily put them into your lab report.

\begin{enumerate}
  \item What break codes does the keyboard send when you type \textbf{key}?
  \item What are lookup tables used in the \textbf{keycode\_to\_ascii.vhd} file?
  \item What is the state of the PS2 clock and data to indicate a idle line?
  \item During you Hardware Acceptance Test Plan on the Keyboard, what keys were incorrect?
  \item Obtain the printouts of your simulation results in you lab report.
  \item Demonstrate the Keyboard Controller with correct outputs.
\end{enumerate}

\newpage

\section{2. VGA Display concepts part 1:}
In this section of the lab you will be learning about the basic concepts of a VGA Display. On the Nexys 2 dev board you have a VGA port with 8-bit color plus two sync signals for a horizontal sync and a vertical sync. With the 8-bit color you are able to create 256 different types of color. The Pin Layout for the VGA can be found on the next figure and in the ucf listed in the VGA Part 1 folder.

\begin{figure}[!htbp]
  \centering
    \fbox{\includegraphics[width=0.4\textwidth]{images/VGALayout.png}}
  \caption{VGA Port Pin Layout (source: \href{http://www.digilentinc.com/}{digilent inc.})}
\end{figure}

\subsection{VGA Concepts}
For a VGA display, signal timing is extremely vital in order to display each pixel to the screen orderly. The VGA display system consist of a horizontal and vertical sync. Both are used to setup the display surface. Further detail on this can be found in \textbf{VGA\_port\_discussion.ppt} under the class dir.

%If you reading this then you can explore the possibility for a VGA explanation section. maybe in v2
% 
% One of the good thing of a standard is that the specifications for VGA can be readily available. If you want more information on the VGA specifications, you can look at \href{http://eewiki.net/x/HgDz}{eewiki}.
% 
% \begin{figure}[!htbp]
%   \centering
%     \fbox{\includegraphics[width=0.5\textwidth]{images/RayGun.png}}
%   \caption{CRT deflection system (source: \href{http://www.digilentinc.com/}{digilent inc.})}
% \end{figure}
% \begin{figure}[!htbp]
%   \centering
%     \fbox{\includegraphics[width=0.5\textwidth]{images/VGADisplay.png}}
%   \caption{VGA System Signals (source: \href{http://www.digilentinc.com/}{digilent inc.})}
% \end{figure}
% 
% \begin{figure}[!htbp]
%   \centering
%     \fbox{\includegraphics[width=0.5\textwidth]{images/VGAControllerCircuit.png}}
%   \caption{VGA Controller Schematic (source: \href{http://www.digilentinc.com/}{digilent inc.})}
% \end{figure}

\subsection{Testing the VGA RGB Displayer}
For this section you will be building the RGB Color project. After you have built and flash the Nexys 2. Test out the displayer unit and change the color to each group members favorite. Save your color code so you can implement it in the VGA RGB file when you are on the next section.

\begin{figure}[!htbp]
  \centering
    \fbox{\includegraphics[width=0.5\textwidth]{images/VGAColorRTL.png}}
  \caption{VGA Color RTL Diagram}
\end{figure}

\subsection{VGA Part 2: Objectives}

\begin{enumerate}
  \item Demonstrate to the TA or Professor the RGB Display Terminal
  \item What is the resolution of the VGA Display in the RGB Display Terminal?
  \item What is the purpose of the \textbf{hcount} and \textbf{vcount} in the VGA Controller?
\end{enumerate}

\newpage

\section{3. VGA Display concepts part 2:}
In this section, you will be building from the previous section on how a VGA Display works. You will be using this concept and the example code provided to create a VGA Display with a ASCII Terminal. During this section you will be learning how to generate memory through Xilinx\textsuperscript{\textregistered} logicore utility. This will allow you to use the memory provided on the FPGA as a buffer for you terminal.

\subsection{VGA Setup}
Create a new project or empty the files in the current file. Include all the files in the VGA Part 2 folder and the Keyboard folder. When you open up the design view, you will notice a file missing(U7). U7 is the memory for the VGA. This file you will be needing to generate through logicore in the next subsection.

\subsection{Generate Memory}
To Generate the memory for the VGA Display, you first need to open up logicore. This can be done under \textbf{Tools} -\textgreater \textbf{Core Generator} or Alt+t,c. When the Core Generator opens up, you will now need to create a new project.

\begin{enumerate}
  \item Enter the parameters for the new project: \hfill \\
  Family : \textbf{Spartan3E}\\
  Device : \textbf{XC3S500E}\\
  Package : \textbf{FG320}\\
  Speed Grade : \textbf{-4}\\
\end{enumerate}

The next part you need to do is generate the memory with the \textbf{Block Memory Generator}. This can be done by going in the IP Catalog and selecting \textbf{Memories \& Storage Elements} -\textgreater \textbf{RAMSs \& ROMs} -\textgreater \textbf{Block Memory Generator}. For the Memory you will need to name it \textbf{VGA\_BUFFER\_RAM}. The memory will be a simple Dual Port Ram with \textbf{8} bit width while having \textbf{4096} depth. On page 4 you will need to initialize the memory. This can be done by loading the coe file in the VGA Part 2 folder. I recommend filling in the remaining memory locations with spaces(\textbf{X"20"}). Follow the next four figures to double check the configurations. After you finished with the configuration, the core generator will generate the xco files you need for the project. Don't forget to check your design properties for the preferred language \textbf{VHDL}. 

\begin{figure}[!htbp]
  \centering
    \fbox{\includegraphics[width=0.5\textwidth]{images/GenMem1.png}}
  \caption{Generate Memory Wizard Page 1}
\end{figure}

\begin{figure}[!htbp]
  \centering
    \fbox{\includegraphics[width=0.5\textwidth]{images/GenMem2.png}}
  \caption{Generate Memory Wizard Page 2}
\end{figure}

\begin{figure}[!htbp]
  \centering
    \fbox{\includegraphics[width=0.5\textwidth]{images/GenMem3.png}}
  \caption{Generate Memory Wizard Page 3}
\end{figure}

\begin{figure}[!htbp]
  \centering
    \fbox{\includegraphics[width=0.5\textwidth]{images/GenMem4.png}}
  \caption{Generate Memory Wizard Page 4}
\end{figure}

\newpage
\subsection{Testing the VGA Display}
After you generate the Logicore file for the memory. The next step you need to do is test the VGA display. First you need to generate the bit file and flash it to the Nexys 2. After you flash it, test how the terminal works. If you want, change the display controls to suite you.

\subsection{VGA Part 2: Objectives}

\begin{enumerate}
  \item Demonstrate to the TA or Professor the ASCII Display Terminal
  \item Why is a 8 to 1 MUX needed?
  \item What is the reason for a blinker?
\end{enumerate}

\newpage

\section{4. Debug Unit:}
With the Previous Sections of the lab, you learned how a finite state machine can be used to grab data. How you can store that data in memory and call it back at a later time. For this section you will be given a design on a potential debug unit. You will be using this design and implementing it in VHDL.

\section{Debug Unit Concepts and Design}
For the Debug Unit, you will be having it monitor the ASCII line. It will check if the ASCII line have been updated with the \textbf{ASCII\_RD} signal from the Keyboard Controller. Each time the \textbf{ASCII\_RD} Line goes high, you will move the ASCII value on the bus to a buffer and increment the position of the buffer. When you hit enter, the debug unit will then flush its contents out for display.

The Debug Unit buffer will output to three devices which will watch the buffer. For Example when you type \textbf{SUB}, the opcode will look through a certain blocks on the buffer and output a binary 1. This can be done with a lookup table. The same principle can be adapted for the registers with the need to sanities the input and only have numbers output and this can be done with another lookup table. You are free to choose what method you want to create the Debug Unit. For this section, you will need to create a test bench and test you debug unit. After testing you will move to the next section of the lab.

\begin{figure}[!htbp]
  \centering
    \fbox{\includegraphics[width=0.5\textwidth]{images/DebugUnitLayout.png}}
  \caption{Basic Layout for the Debug Unit}
\end{figure}

\subsection{Debug Unit: Objectives}
\begin{enumerate}
  \item Develop a block diagram of your Debug Unit and include it in the report.
  \item Implement the design in VHDL.
  \item Create a Test Bench to test your design
  \item Capture you simulation output to verify it is working.
\end{enumerate}

\newpage

\section{5. Assembling a Keyboard Debug Unit:}
After Learning about the concepts of a debug unit, you will now need to add a keyboard with the debug unit. You will need to have the keyboard controller ASCII line linked up with the debug unit with the output data to the I/O's on the board. If you want a challenge, implement the debug unit with an output to the VGA display, this can be used later on for Extra Credit Part 2(optional).

\subsection{Keyboard Debug Unit: Objectives}
\begin{enumerate}
  \item Develop a block diagram of you design include it in your report.
  \item Implement the design in VHDL.
  \item Demonstrate the Keyboard with Debug Unit to the TA or professor.
\end{enumerate}

%\newpage %Save the Treesssss one page at a time

\section{6. Extra 1: Keyboard ALU part 1:}
With the experience you learned throughout each section of the Lab. You will now be combining your knowledge for some extra credit. %or 12 years dungeon, all of you dungeon, 7 years no trial.
You will be having the debug unit control the ALU and have it output data on the 7-segment display. You will need to have the input from a keyboard. This will be very useful during the project since you will be able to enter commands from the keyboard to your device. In order to gain extra credit, you must complete each objective. %not kidding, you need to complete everything....

\subsection{Extra Credit 1: Objectives}
\begin{enumerate}
  \item Develop a block diagram of you design include it in your report.
  \item Implement the design in VHDL.
  \item Demonstrate the Keyboard to ALU to the TA or professor.
\end{enumerate}

\section{7. Extra 2: Keyboard ALU part 2:}
%Time for the REAL Extra credit :)
For the next extra credit you will have the ALU output directly to the VGA terminal. You need to build upon from the previous extra credit and change the output. To get started, you will be needing to manipulate the ASCII bus to have the ALU send to the VGA Terminal.

\subsection{Extra Credit 2: Objectives}
\begin{enumerate}
  \item Develop a block diagram of you design include it in your report.
  \item Implement the design in VHDL.
  \item Demonstrate the Keyboard to ALU to the TA or professor.
\end{enumerate}


\newpage

\section{Lab 2 Grade}

\begin{table}[!htb]
  \begin{center}
    \begin{tabular}[width=0.8\textwidth]{|l|l|c|l|}
       \hline
       Section & Description & Value & Score\\
       \hline 
       \multicolumn{2}{|l}{\textbf{Section 1}}  & -\textbf{30}- &\\
       \hline
       1.1. Keyboard Demo & Demonstrate Keyboard Controller & 20 &\\
       1.2. Questions & Answer Questions about the Keyboard Controller & 10 &\\
       \hline
       \multicolumn{2}{|l}{\textbf{Section 2}}  & -\textbf{20}- &\\
       \hline
       2.1. VGA Demo 1 & Demonstrate RGB VGA Display & 10 &\\
       2.2. Questions & Answer Questions about the VGA Display & 10 &\\
       \hline
       \multicolumn{2}{|l}{\textbf{Section 3}}  & -\textbf{10}- &\\
       \hline
       3.1. VGA Demo 2 & Demonstrate VGA Terminal Display & 5 &\\
       3.2. Questions & Answer Questions about the VGA Terminal Display & 5 &\\
       \hline
       \multicolumn{2}{|l}{\textbf{Section 4}}  & -\textbf{10}- &\\
       \hline
       4.1. Debug Unit Demo &  Demonstrate Debug Unit & 5 &\\
       4.2. Questions & Answer Questions about the Debug Unit & 5 &\\
       \hline
       \multicolumn{2}{|l}{\textbf{Section 5}}  & -\textbf{10}- &\\
       \hline
       5.1. VGA Debug Unit & Demonstrate VGA with Debug Unit & 5 &\\
       5.2. Questions & Answer Questions about the VGA Debug Unit & 5 &\\
       \hline
       \multicolumn{2}{|l}{\textbf{Section 6: Extra Credit 1}}  & -\textbf{20}- &\\
       \hline
       6.1. ALU Debug Unit & Demonstrate Debug Unit with ALU & 20 &\\
       \hline
       \multicolumn{2}{|l}{\textbf{Section 7: Extra Credit 2}}  & -\textbf{20}- &\\
       \hline
       7.1. ALU Terminal & Demonstrate ALU with VGA Terminal & 20 &\\
       \hline
       \multicolumn{2}{|l}{\textbf{Lab Report}}  & -\textbf{20}- &\\
       \hline
       \hline
       \multicolumn{2}{|l}{Total} & \multicolumn{1}{c|}{100|140} &\\
       \hline
    \end{tabular}
  \end{center}
  \caption{Lab Grade Breakdown Table}
\end{table}

\end{document}
