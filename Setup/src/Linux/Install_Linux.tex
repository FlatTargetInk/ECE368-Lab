\documentclass[letter]{article}
\usepackage{listings}
\usepackage[margin=1in]{geometry}
\usepackage{xcolor}
\usepackage{graphicx}
\usepackage{textcomp}
\usepackage{hyperref}

\begin{document}

\title{Xilinx Linux Installation Guide}
\author{Daniel Noyes}
\date{\today}
\maketitle
\lstset{showstringspaces=false, basicstyle=\ttfamily,frame=tb,columns=fullflexible, 
	stringstyle=\color{green},
	commentstyle=\color{gray},
	keywordstyle=\color{blue},
	}

\section{Introduction}
The Installation for Xilinx is unique based on what linux distribution you are using. In the setup folder, their are two bash scripts that will help with the installation. These scripts will be talked about in further detail later on in this Guide. The first step you need to do is to download Xilinx ISE 14.7. Inorder to download Xilinx ISE, you first must register on Xilinx website. After registering, on the download page you will see multiple different formats to download from. This guide will use the "Multi-File Download" since this will help break down the application in reasonable chunks to download.

\begin{center}
	\href{http://www.xilinx.com/support/download/index.html/content/xilinx/en/downloadNav/design-tools.html}{Xilinx ISE Design Edition Download Page}
\end{center}

\begin{figure}[!htbp]
  \centering
    \fbox{\includegraphics[width=0.8\textwidth]{images/XilinxDownloadPage.png}}
  \caption{\texttrademark Xilinx Download Page}
\end{figure}

\section{Preperation}
Once you downloaded each file necessary for the desired download format. It is recomended to validate the contents of each file. This can be done in the terminal with the command "md5sum", you will need to check the value with the one one the download page. As I said, this is optional but recomended. You now then need to extract the "Split installer Base Image"(Xilinx\_ISE\_DS\_14.7\_1015\_1-1.tar) into a seperate temporary directory. This can be done with your current distributions archive manager.

\begin{figure}[!htbp]
  \centering
    \fbox{\includegraphics[width=0.8\textwidth]{images/Extract1.png}}
  \caption{Archive Manager Selection}
\end{figure}


\section{Installation}
After you extract the installer, their will be a "xsetup" bash script. Run it to initilize the Xilinx installer. The installer has 7 steps to it which are very straight forward. If you need any assistances, follow the next few sections of the installation. If you complete the Installation alread, head to the Finilization section.

\subsection{Welcome}
Welcome to the installation of ISE, their will be 7 steps to this. Dont worry they are fairly easy to complete. For the first page just hit next.

\begin{figure}[!htbp]
  \centering
    \fbox{\includegraphics[width=0.8\textwidth]{images/Installer1.png}}
  \caption{Welcome to the ISE Installer}
\end{figure}

\subsection{Select Download Location}
The ISE installer will be needing the other 3 files that your downloaded from xilinx. Just point this to the location where you save them. If the screen goes grey and nothing is happening, the installer gui crashed while trying to grab the qt filebrowser. You have to \^C the terminal where xsetup is running and just fill in the locations manually instead of using browse.

\begin{figure}[!htbp]
  \centering
    \fbox{\includegraphics[width=0.8\textwidth]{images/Installer2.png}}
  \caption{Example of a Download Location Directory for the additional installation files}
\end{figure}

\subsection{Accept the Licensing Agreement}
There are two license Agreements for ISE, feel free to read all of it. After you finish reading, accept the terms and move on to the next step

\begin{figure}[!htbp]
  \centering
    \fbox{\includegraphics[width=0.8\textwidth]{images/Installer3.png}}
  \caption{License Agreements}
\end{figure}

\subsection{Select the Product to Install}
Out of the many different versions of the produce. We will be selection the "ISE WebPACK" edition. Make sure you downloaded the webpack license from xilinx, you will be needing it when the application first starts.

\begin{figure}[!htbp]
  \centering
    \fbox{\includegraphics[width=0.8\textwidth]{images/Installer4.png}}
  \caption{Select ISE WebPACK Edition}
\end{figure}

\subsection{Select Installation Options}
This step is cruitial for the installation. You will be needing to install the license manager to maintain your webpack license. It is recomended to install the symlinks to make it easyer to maintain xilinx on your system. The Install Cable Drivers is optional but I dont recomend it. Their is a xilinx driver install script you should use instead.

\begin{figure}[!htbp]
  \centering
    \fbox{\includegraphics[width=0.8\textwidth]{images/Installer5.png}}
  \caption{Installation Options}
\end{figure}

\subsection{Select a Installation Directory}
Fairly straight forward, for xilinx it is recomended to install in the /opt folder on you system.

\begin{figure}[!htbp]
  \centering
    \fbox{\includegraphics[width=0.8\textwidth]{images/Installer6.png}}
  \caption{Installation Options}
\end{figure}

\subsection{Installation}
The next step is to finalize your settings. Next you will have to wait for it to install and your done with the installation.

\begin{figure}[!htbp]
  \centering
    \fbox{\includegraphics[width=0.8\textwidth]{images/Installer7.png}}
  \caption{Installation Summary}
\end{figure}

\newpage
\section{Finilization}
After you completed the installation of Xilinx their is only a few more steps to go till you finish setting it up. The last two steps are to install the drivers for your system and to add the application icons.

\subsection{Xilinx Driver Install}
Inorder to isntall the xilinx drivers you just have to run the "Xilinx\_driver\_install.sh" script. Inorder to install you just have to type the following "Xilinx\_driver\_install.sh install /opt/Xilinx/14.7/". Just remember to include up to the 14.7 or else it will not work.

\begin{lstlisting}[language=bash,caption={Xilinx\_driver\_install.sh example}]
root@dnoyes ~ # ./xilinx_driver_install.sh /opt/Xilinx/14.7
\end{lstlisting}

\subsection{Xilinx Icon Install}
Now the last step, install application icons to make it easy to use. Just run "Xilinx\_icon\_install.sh install /opt/Xilinx/14.7/" and your done.

\begin{lstlisting}[language=bash,caption={Xilinx\_icon\_install.sh example}]
root@dnoyes ~ # ./xilinx_icon_install.sh /opt/Xilinx/14.7
\end{lstlisting}

\section{First Time Start}
Now lets test to see if the installation is complete. open you application folder and go to Development. Under Development you will se ise and xps. Click on ise and wait for it to load. If it opens up, congradulations you have a installed version of Xilinx.

\begin{figure}[!htbp]
  \centering
    \fbox{\includegraphics[width=0.6\textwidth]{images/Startup.png}}
  \caption{Start ISE}
\end{figure}

\end{document}
