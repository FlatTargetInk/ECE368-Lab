\documentclass[letter]{article}
\usepackage{listings}
\usepackage[margin=1in]{geometry}
\usepackage{xcolor}
\usepackage{graphicx}
\usepackage{textcomp}
\usepackage{hyperref}

\begin{document}

\title{Xilinx Windows Installation Guide\\v1.0}
\author{Daniel Noyes}
\date{\today}
\maketitle

\section{Introduction}
The Installation for Xilinx is based on Windows 7 and can be used on Windows 8 and 10. The first step you need to do is to download Xilinx ISE 14.7. In order to download Xilinx ISE, you first must register on Xilinx website. After registering, on the download page you will see multiple different formats to download from. This guide will use the "Multi-File Download" since this will help break down the application in reasonable chunks to download.

\begin{center}
	\href{http://www.xilinx.com/support/download/index.html/content/xilinx/en/downloadNav/design-tools.html}{Xilinx ISE Design Edition Download Page}
\end{center}

\begin{figure}[!htbp]
  \centering
    \fbox{\includegraphics[width=0.8\textwidth]{images/XilinxDownloadPage.png}}
  \caption{\texttrademark Xilinx Download Page}
\end{figure}

\section{Preparation}
Once you downloaded each file necessary for the desired download format. It is recommended to validate the contents of each file. For windows unfortiently their is no inbedded application readily available and you have to download a 3rd party application like \href{http://www.md5summer.org/}{MD5summer}. You will need to check the MD5sum values on the download page. As I said, this is optional but recommended. You now then need to extract the "Split installer Base Image"(Xilinx\_ISE\_DS\_14.7\_1015\_1-1.tar) into a separate temporary directory. This can be done with you choice of decompressions applications like 7zip or winrar.

\begin{figure}[!htbp]
  \centering
    \fbox{\includegraphics[width=0.8\textwidth]{images/Extract1.png}}
  \caption{7zip Extract Selection Example}
\end{figure}


\section{Installation}
After you extract the installer, their will be a "xsetup" appilication application in it. Run it to initialize the Xilinx installer. The installer has 8 steps to it which are very straight forward. If you need any assistances, follow the next few sections of the installation. If you complete the Installation already, head to the First Time Start section.

\begin{figure}[!htbp]
  \centering
    \fbox{\includegraphics[width=0.8\textwidth]{images/InstallerSelect1.png}}
  \caption{xsetup location}
\end{figure}

\subsection{Welcome}
Welcome to the installation of ISE, their will be 8 steps to this. Don't worry they are fairly easy to complete. For the first page just hit next.

\begin{figure}[!htbp]
  \centering
    \fbox{\includegraphics[width=0.8\textwidth]{images/Installer1.png}}
  \caption{Welcome to the ISE Installer}
\end{figure}

\subsection{Select Download Location}
The ISE installer will be needing the other 3 files that your downloaded from xilinx. Just point this to the location where you save them. The Installer will normally point to the Directory where it is located.

\begin{figure}[!htbp]
  \centering
    \fbox{\includegraphics[width=0.8\textwidth]{images/Installer2.png}}
  \caption{Example of a Download Location Directory for the additional installation files}
\end{figure}

\subsection{Accept the Licensing Agreement}
There are two license Agreements for ISE, feel free to read all of it. After you finish reading, accept the terms and move on to the next step

\begin{figure}[!htbp]
  \centering
    \fbox{\includegraphics[width=0.8\textwidth]{images/Installer3.png}}
  \caption{License Agreements}
\end{figure}

\subsection{Select the Product to Install}
Out of the many different versions of the produce. We will be selection the "ISE WebPACK" edition. Make sure you downloaded the webpack license from xilinx, you will be needing it when the application first starts.

\begin{figure}[!htbp]
  \centering
    \fbox{\includegraphics[width=0.8\textwidth]{images/Installer4.png}}
  \caption{Select ISE WebPACK Edition}
\end{figure}

\subsection{Select Installation Options}
This step is crucial for the installation. You will be needing to install the license manager to maintain your webpack license. It is recommended to install both WinPcap and the Cable Drivers.

\begin{figure}[!htbp]
  \centering
    \fbox{\includegraphics[width=0.8\textwidth]{images/Installer5.png}}
  \caption{Installation Options}
\end{figure}

\subsection{Select a Installation Directory}
Fairly straight forward, for xilinx it is recommended to install in the root of the C: folder on you system but it can go anywhere on your system.

\begin{figure}[!htbp]
  \centering
    \fbox{\includegraphics[width=0.8\textwidth]{images/Installer6.png}}
  \caption{Installation Options}
\end{figure}

\subsection{Installation}
The next step is to finalize your settings. Next you will have to wait for it to install and your done with the installation.

\begin{figure}[!htbp]
  \centering
    \fbox{\includegraphics[width=0.8\textwidth]{images/Installer7.png}}
  \caption{Installation Summary}
\end{figure}

\subsection{}
During the installation process, the installer will ask you to install WinPCap and you will be asked to install drivers. 

\begin{figure}[!htbp]
  \centering
    \fbox{\includegraphics[width=0.8\textwidth]{images/WinPcap1.png}}
  \caption{WinPCap Install}
\end{figure}

\begin{figure}[!htbp]
  \centering
    \fbox{
	\includegraphics[width=0.8\textwidth]{images/Driver1.png}
	}
  \caption{Jungo Driver install}
\end{figure}

\begin{figure}[!htbp]
  \centering
    \fbox{
	\includegraphics[width=0.8\textwidth]{images/Driver2.png}
	}
  \caption{Xilinx Driver install}
\end{figure}

\newpage

\section{First Time Start}
Now lets test to see if the installation is complete. open xilinx project explorer either from your desktop or you Icons folder. Upon first time loading you will be prompted with a License manager page, you will need to supply the WebPack license from Xilinx license page. After that you are now have a fully installed xilinx application.

\begin{figure}[!htbp]
  \centering
    \fbox{\includegraphics[width=0.6\textwidth]{images/License1.png}}
  \caption{Startup License Manager}
\end{figure}

\begin{figure}[!htbp]
  \centering
    \fbox{\includegraphics[width=0.6\textwidth]{images/Startup.png}}
  \caption{Start ISE}
\end{figure}

\end{document}
